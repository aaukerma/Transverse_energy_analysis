\chapter{Introduction} \label{ch:introduction}

%\section{A Brief History of the Universe}
The Big Bang model is based on observational evidence, such as the cosmic microwave background radiation and the cosmological expansion, and suggests that at the beginning the universe must have been at a state of really high density and temperature. As the universe expanded, it went through several stages of cooling characterized by the formation of matters with different compositions. The matter we mostly observe today exists at temperatures and densities much lower compared to those in the early universe.

%\section{Production of Historical Matter}
The Large Hadron Collider (LHC) at CERN and the Relativistic Heavy Ion Collider (RHIC) at the Brookhaven National Laboratory have the ability to collide heavy nuclei, such as those of gold and uranium, at nearly the speed of light, reaching temperatures of trillions of degrees Celcius. These laboratories have provided evidence of the formation of an exotic state of matter, called the quark-gluon plasma (QGP). It only exists for a brief amount of time after such collisions and instantly freezes out into a plethora of new particles, which carry the signatures we can use to deduct QGP properties. Its properties suggest that it should be similar to the matter that existed within microseconds of the genesis of the universe. It behaves like an almost perfect fluid with a viscosity near 0 \cite{RHIC white paper}.

%\section{Motivation of This Thesis}
One of the methods to probe the properties of this matter is by analyzing the conversion of the beam-direction energy at the time of collision into transverse energy after the collision. These measurements can be used to estimate the energy density of the QGP. This analysis is generally done by using data from the calorimeters placed around the collision site. In this thesis, I use the data collected by tracking detectors, instead of the conventional calorimeters, to calculate the transverse energy.

%\section{Organization of The Thesis}
This thesis is structured as follows. chapter \ref{ch:background} touches on the theoretial background associated with the concept of the quark-gluon plasma. In chapter \ref{ch:RHI-collisions}, I summarize the experimental concepts pertaining to relativistic heavy-ion collisions and the production and detection of QGP. chapter \ref{ch:measurement} consists of the formalism of the measurement of transverse energy using calorimeters as well as tracking detectors. It also describes what has been done using calorimeters. chapter \ref{ch:analysis} describes the data used to perform the analysis in this thesis and notes the relevant details of the analysis. In chapter \ref{ch:results}, I present the results and compare them to the ones in literature obtained using a different method. Chapter \ref(ch:conclusion) concludes the thesis and discusses its implications. Finally, in chapter \ref{ch:future}, I present arguments on what can be done in the future using the results of and the software developed for this analysis.
