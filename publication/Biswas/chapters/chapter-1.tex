\chapter{Introduction} \label{ch:introduction}

The Large Hadron Collider (LHC) at CERN and the Relativistic Heavy Ion Collider (RHIC) at the Brookhaven National Laboratory have the ability to collide heavy nuclei, such as those of gold and uranium, at nearly the speed of light, reaching temperatures of trillions of degrees Celcius. These laboratories have provided evidence of the formation of an exotic state of matter, called the quark-gluon plasma (QGP). It only exists for a brief amount of time after such collisions and instantly freezes out into a plethora of new particles, which carry the signatures we can use to deduct QGP properties. It reportedly behaves like an almost perfect quantum fluid with no resistance and exhibits other interesting properties.

One of the methods to probe the properties of this matter is by analyzing the conversion of the beam-direction energy at the time of collision into transverse energy after the collision. This analysis is generally done by using data from the calorimeters placed around the collision site. In this thesis, I use the data collected by the tracking detectors, instead of the conventional calorimeters, to perform the transverse energy analysis.

The organization of the thesis is as follows. In Chapter 2, I attempt to summarize the physical concepts pertaining to nuclear matter, heavy-ion collisions, and the production and detection of QGP. Chapter 3 consists of the formalism of the measurement of transverse energy using calorimeters as well as tracking detectors. It also gives an example of what has been done using calorimeters. Chapter 4 describes the data used to perform the analysis in this thesis and notes down the details of the analysis. In Chapter 5, I present the results and compare them to the ones in literature obtained using a different method. Chapter 6 concludes the thesis by summarizing it and shedding light on some of its implications.
