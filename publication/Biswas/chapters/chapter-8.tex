\chapter{Future Work}\label{ch:future}

\section{Goodness of Fit}
A maximum likelihood? fit method can be adopted to compare the results with those using the chi-squared fits.

\section{Bjorken Energy Density Estimate}
Apart from the transverse energy, the calculation of the initial energy density, $\epsilon$, as given by the Bjorken formula in eq. \ref{eqn:bjorken}, requires the estimate of other physical quantities. Adare et al.\cite{PhysRevC.93.024901} use the Glauber model to determine $A_{T}$, the area of the intersection of the two nuclei in the transverse plane. Since the results in this thesis are cross-checked with those in \cite{PhysRevC.93.024901}, it would be reasonable to use the same model in the future work pertaining to this thesis. $\tau_{0}$, the proper time at the moment of QGP equilibration, also depends on the model of the collision. However, the product of $\epsilon$ and $\tau_{0}$ is often used instead of just $\epsilon$ to study how the energy density scales with the collision energy and the number of participants.

\section{Asymmetric beams}
The codes in the repository can be used to analyze more data. In fact, since there is more data available on collisions of asymmetric systems such as d+Au?, we can expect it to be a test to tell if the assumptions? used in this analysis scale to such domains?

