\chapter{Introduction} /label{ch:introduction}

One of the main focues of current experimental and theoretical nuclear physics research is the study of the phase diagram of nuclear matter at a range of temperatures and baryon chemical potentials. In experiments involving the collisions of heavy ions at relativistic energies, different regions of the phase diagram can be probed by varying the collision energy (1509.06727: ALICE 2013). For instance, the high-baryon chemical potential regime corresponds to lower beam energies. Theories point toward the phase transition, at energy densities above 0.2-1 GeV/fm^3 (alice 2016 CERN-EP-2016-071), of nuclear matter to a phase with partons representing the relevant degrees of freedom and a possible transition of that matter into a phase in which hadrons represent the relevant degrees of freedom. Observations made in experiments involving collisions of heavy ions at high energies have provided signatures of the formation of matter with partonic degrees of freedom at the early stages of the collisions. Such signatures include a diminished multiplicity of high-transverse momentum hardrons in heavy ion collisions as compared to scaled p-p collisions as well as the differences bewtween the elliptic flows associated with baryons and mesons at intermediate pT in heavy-ion collisions (BES paper).
