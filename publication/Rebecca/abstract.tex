\chapter*{Abstract}\label{ch:abstract}


%%Moved to the Introduction
%Ultrarelativistic nucleus-nucleus collisions at the Large Hadron Collider are used to study nuclear matter under high temperature and energy density by creating a Quark Gluon Plasma (QGP). In heavy ion collisions, heavy quarks, charm and beauty, are created early in the collision and traverse the entire evolution thus making them a good probe of the medium. 
%
%Electrons from semileptonic decays of charm and beauty hadrons provide one way of measuring heavy flavor production in p-Pb collisions. The measurement of non-photonic electrons is representative of the measurement of heavy quarks since the spectrum is dominated by electrons from the semileptonic decays of D and B mesons. 
%
%The interaction with heavy flavor and the medium can be studied with the production of electrons from heavy flavor decays from Pb-Pb collisions as compared to p-p collisions. Complementary studies of p-Pb collisions were meant to isolate cold nuclear matter effects and mechanisms unrelated to the presence of a QGP. However there are some recent hints that there may also be collective effects in p-Pb collisions making the separation of cold and hot nuclear matter effects more challenging. Initial and final state effects related to the presence of cold nuclear matter can affect the heavy-flavor yield in p-Pb collisions. 





%The main goal of the ALICE detector at the Large Hadron Collider is to study the quark gluon plasma that is created in the ultra relativistic nucleus-nucleus collisions. Complementary studies of proton-proton collisions and proton-nucleus collisions are not expected to produce a quark-gluon plasma and are essential to provide a baseline for the measurements carried out in heavy-ion collisions. Proton-nucleus collisions are also interesting in their own right, providing a view into cold nuclear matter effects and the modifications to parton distributions in the proton and the nucleus.
%
%Heavy flavor quarks, charm and bottom, are produced from hard parton scattering early in the collision and traverse the medium throughout its evolution, making them a good probe. The production of heavy-flavor is sensitive to the parton distributions and the medium produced in the collisions. Due to the large mass of the charm and bottom quark, their production cross sections can be calculated by perturbative Quantum Chromodynamics. Single electrons from the semileptonic decays of charm and bottom mesons provide one way of measuring heavy flavor production.
%
%The main goal of this dissertation is to measure the production of single electrons that come from the decay of heavy flavor D and B mesons. The dominant backgrounds are hadrons misidentified as electrons, and the so-called ``photonic electrons'', which are produced in pairs. 

This dissertation presents a measurement of the yield and cross section of electrons from heavy flavor decays at central rapidity in proton-lead collisions measured by the ALICE (A Large Ion Collider Experiment) detector at the Large Hadron Collider. This analysis extends the transverse momentum reach of an earlier measurement in ALICE and the comparison is shown. The cross section of single electrons in proton-lead collisions is compared to the value expected in the absence of nuclear modification from proton-proton collisions. The cross section is well described by the perturbative Quantum Chromodynamics and no statistically significant alteration due to hot nuclear matter effects is observed.  The results are also compared to other measurements of heavy flavor and collision systems. 
 



%From the APS talk:
%Ultrarelativistic nucleus-nucleus collisions at the Large Hadron Collider are used to study nuclear matter under high temperature and energy density by creating a Quark Gluon Plasma (QGP). Complementary studies of p-Pb collisions were meant to isolate cold nuclear matter effects and mechanisms unrelated to the presence of a QGP. However there are some recent hints that there may also be collective effects in p-Pb collisions making the separation of cold and hot nuclear matter effects more challenging. In Pb-Pb collisions, heavy quarks, charm and beauty, are created early in the collision and traverse the entire evolution thus making them a good probe of the medium. In p-Pb collisions, initial and final state effects related to the presence of cold nuclear matter can affect the heavy-flavor yield. Electrons from semileptonic decays of charm and beauty hadrons provide one way of measuring heavy flavor production in p-Pb collisions. The status of the current analysis of electrons from heavy flavor decays in p-Pb collisions, with particular emphasis on the electron identification, will be presented.

%From my comprehensive:
%Ultrarelativistic nucleus-nucleus collisions produced at the Large Hadron Collider create a Quark Gluon Plasma. Heavy quarks, which are produced from hard parton scattering, are created in the early stages of these collisions. Heavy quarks' interaction with the medium makes them a good probe of the Quark Gluon Plasma. Electrons from semileptonic decays of charm and beauty hadrons provide one way of measuring heavy flavor quarks in heavy ion collisions. I propose to measure non-photonic electrons in ?s = 2.76 TeV lead+lead collisions using the ALICE detector at the LHC. I will compare the non-photonic electron spectrum from lead+lead collisions to the spectrum from proton+proton collisions to determine the energy loss of heavy flavor quarks traveling through the QGP. I will also analyze the angular distribution of electrons with respect to the reaction plane to study flow phenomena at low transverse momenta and path length dependence of energy loss at high transverse momenta. This study will provide a test of theoretical models of the QGP and give a better understanding of how heavy flavor interacts with the QGP.

