\chapter{Summary and Conclusions} \label{ch:conclusions}

The purpose of the measurement described in this work is to address issues concerning cold nuclear matter effects in the production of the heavy flavor quarks in relativistic heavy ion collisions. Is the yield of heavy flavor produced in collisions affected by the additional nucleons present in proton-nucleus collisions as compared to proton-proton collisions? Additionally, is the production and momentum distribution of heavy quarks affected differently from light quarks? Furthermore, are there new physics effects at the high LHC energies as compared to what has been previously measured at RHIC?

In order to address these questions, the Large Hadron Collider accelerated and collided protons at 4 TeV with lead ions at 82x4 TeV. These asymmetrical proton-lead collisions at center of mass energy per nucleon $\sqrt{s_{NN}} = 5.02$ TeV were measured by ALICE. This analysis measured the yield of mesons composed of one charm or bottom quark and a light quark, $D^+ , D^- , D^0 , \bar D^0, B^+ , B^- , B^0 , \bar B^0$. The production of $D$ and $B$ hadrons were measured through their semileptonic decay channel by measuring the single electron yield. 

Electrons were enhanced by using the tracking and energy measurements provided by the Time Projection Chamber and the Electromagnetic Calorimeter. The remaining hadron contamination was subtracted with a function that was fit to the detector response of the electron signal and the hadron background. This measurement covered a novel transverse momentum range made possible by the EMCal trigger system. However, these triggers introduced an additional background that needed to be accounted for in the background subtraction. 

The main sources of background electrons are from photon conversion and from Dalitz decay of light mesons. This analysis estimated the yield of background electrons by exploiting the fact that the main background sources produced electrons in pairs with an invariant mass near zero, while the signal produced single electrons. 

The final results of the yield, cross section, and $R_{pPb}$ of electrons from the decays of $D$ and $B$ mesons in $pPb$ collisions as measured in ALICE at the LHC were presented in figure \ref{fig:HFEYieldAllTriggers}, figure \ref{fig:HFECSCombined} and figure \ref{fig:CombinedRAAFONLL}.

This analysis extended the transverse momentum reach of an earlier measurement in ALICE \cite{Adam:2015qda}. The measurement of the cross section agrees with the earlier measurement as shown in figure \ref{fig:PubAndRebCSCombined}. The transverse momentum coverage of this analysis was able to extend the $p_{T}$ reach of the previous measurement from 12 GeV/c up to 30 GeV/c using the EMCal triggers. 

The $R_{pPb}$ of this measurement demonstrated agreement with the previous measurement in the $p_{T}$ overlap region in figure \ref{fig:RebeccaAndJanRAA}. A similar analysis, that reconstructs the D meson using the hadronic decay channel, was also shown to agree with the measurement from this analysis in figure \ref{fig:Dmesons}.

Theoretically the data are well described by FONLL, Fixed Order plus Next-to-Leading-Log perturbative Quantum Chromodynamics theory calculation with standard parameters. The pQCD theory was calculated for proton-proton collisions and then scaled using the average number of binary nucleon-nucleon collisions to compare with proton-lead collisions illustrated in figure \ref{fig:CombinedCSandScaledFONLL}. FONLL describes the data well across six orders of magnitude. More advanced models that take into account various cold nuclear matter effects were also consistent with this measurement, as depicted in figure \ref{fig:RebeccaAndJanRAA}.

This measurement was compared a similar collision system, $d+Au$, at a lower energy, $\sqrt{s_{NN}} = 200$ GeV measured at RHIC. Figure \ref{fig:DAuPhenixAndMineCombined} shows a slight possible Cronin enhancement at low $p_{T}$ and is consistent with unity at higher $p_{T}$. These results show that the effects of cold nuclear matter in $pPb$ collisions as compared to $pp$ collisions are small for heavy flavor hadrons at LHC energies and do not deviate significantly from the measurements at RHIC energies.

As seen in figure \ref{fig:PionsKaonsProtons}, the $D$ and $B$ mesons seem to scale similarly to other light flavored charged hadrons when comparing $pp$ to $pPb$ collisions. The modification of light and heavy quarks are comparable in $pPb$ collisions.

Figure \ref{fig:DAuPhenixAndMineCombined} and figure \ref{fig:2016-Sep-23-hfe_00_10RHIC} illustrate a significant suppression in heavy ion collisions, $Au+Au$ and the $Pb-Pb$, at RHIC and LHC energies. Since the $d+Au$ and $pPb$ collision data does not show any signs of suppression on heavy flavor, initial state effects and cold nuclear matter effects can not explain the suppression seen in central heavy ion collisions and must be due to hot nuclear matter effects.






